\documentclass[10pt,a4paper,ngerman]{scrartcl}
\usepackage[utf8]{inputenc}
\usepackage[ngerman]{babel}
\usepackage[T1]{fontenc}
\usepackage{amsmath}
\usepackage{amsfonts}
\usepackage{amssymb}
\usepackage{makeidx}
\usepackage{graphicx}
\usepackage{epstopdf}
\usepackage[onehalfspacing]{setspace}
\usepackage{array}
\usepackage{upgreek}
\usepackage{float}
\usepackage{csquotes}
\usepackage{chemformula}
\usepackage{chemgreek}
\usepackage{chemmacros}
\chemsetup{modules=all}
\usepackage{tikz}
\usepackage{siunitx}
\sisetup{locale = DE,per-mode = symbol}
\usepackage{esvect}
\usepackage{geometry}
\usepackage{pdfpages}
\usepackage[font=small]{caption}
%\usepackage[version=4]{mhchem}
\usepackage[backend=biber, style=chem-angew]{biblatex}
\addbibresource{Sheet07.bib}
\usepackage{tabularx, booktabs, multirow}
\usepackage[pdfborder={0 0 0}]{hyperref}
\usepackage{scrlayer-scrpage}%Header für die KOMA-script -Klasse
%\usepackage{hyperref}
\usepackage{pgfplots}
\usepackage{textgreek}
\usepackage{minted}

\newcommand{\tildenu}{{\fontencoding{LGR}\selectfont\accperispomeni\textnu}}
\captionsetup{format=plain}
\chemsetup[phases]{pos=sub}
\chemsetup[redox]{pos=top}
\chemsetup[redox]{align=center}
%\chemsetup[redox]{explicit-sign=true}
%\chemsetup[redox]{roman=false}
\newcommand{\celsius}{^{\circ}\mathrm{C}}
\parindent0pt
\sloppy
\DeclareChemPhase{\s}{s}
\DeclareChemPhase{\l}{l}
\DeclareChemPhase{\g}{g}
\renewcaptionname{ngerman}{\figurename}{Abb.}
\renewcaptionname{ngerman}{\tablename}{Tab.}
\geometry{bottom=100pt} \geometry{top=80pt}
\newcommand{\m}{\mathrm{m}}
\newcommand{\K}{\mathrm{K}}
\newcommand{\J}{\mathrm{J}}
\newcommand{\gr}{\mathrm{g}}
\newcommand{\kg}{\mathrm{kg}}
\newcommand{\mol}{\mathrm{mol}}
\newcommand{\Pa}{\mathrm{Pa}}
\newcommand{\ad}{\mathrm{ad}}
\newcommand{\de}{\mathrm{de}}
\newcommand{\gmol}{\mathrm{\frac{g}{mol}}}
\newcommand{\JKmol}{\mathrm{\frac{J}{K \cdot mol}}}
\newcommand{\kgmss}{\mathrm{\frac{kg}{m \cdot s^2}}}
\newcommand{\loge}{\mathrm{ln}}

\begin{document}
\begin{titlepage}
	\vspace*{2,5cm}
	\begin{center}
		\begin{LARGE}
			\textbf{Physikalische Chemie 2}\\
			\textbf{Elektron-Spin-Resonanz-Spektroskopie}\\
		\end{LARGE}
		\vspace{1cm}
		\textbf{Universität Stuttgart}\\
		\vspace*{1,5cm}
		\begin{tabular}{lp{7,5cm}l}
			Author 1:     &Laurens Viehoff\\
            Author 2:     &Leander Haase\\   
			E-Mail 1:        &st183688@stud.uni-stuttgart.de\\
            E-Mail 2:        &st189365@stud.uni-stuttgart.de  \\
			Student ID 1: & 3676761   \\ 
            Student-ID 2: & 3743225 \\ \\
			Assistentin:  &  Claudia Franke \\ \\ 
		\end{tabular}
	\end{center}

    \begin{center}
        \textbf{\Large Abstract}
    \end{center}
    
\end{titlepage}
\thispagestyle{empty}
\renewcommand{\thepage}{\arabic{page}}
\newpage
\tableofcontents
\setcounter{page}{1}
\renewcommand{\thepage}{\arabic{page}}
\newpage

\section{Aufgabe 1}
Hier der Nachweis für die Nature einbindung von Quelle \cite{biamonte_quantum_2017}.
\begin{figure}[H]
	\centering
    \includegraphics[width=0.9\linewidth]{Zotero_Nachweis_quantum.png}
    \caption{Hier des Quantum machine learning paper.}
    \label{fig:Quantummachinelearning}
\end{figure}
Hier dann noch der Nachweis der eigenen Erstellung einer Quelle vom Buch \cite{demtroder_experimentalphysik_2003}.
\begin{figure}[H]
	\centering
    \includegraphics[width=0.9\linewidth]{Zotero_Nachweis_Demtröder.png}
    \caption{Hier das Demtröder Buch.}
    \label{fig:Raman_Chloroform}
\end{figure}


\section{Aufgabe 2}
\begin{minted}[]{python}
def is_even(number):
"""Check, if a given number is even.
...
"""
    assert isinstance(number, int) and number >= 1, 
	"Die gegebene Zahl ist keine positive, natürliche Zahl."
    return number % 2 == 0

assert not is_even(9)
assert is_even(10)

print(f"is_even(9): {is_even(9)}")
print(f"is_even(10): {is_even(10)}")
\end{minted}




\printbibliography

\end{document}