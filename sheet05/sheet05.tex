\documentclass[a4paper,12pt,bibliography=totocnumbered]{scrartcl}
\usepackage[utf8]{inputenc} %Codierung des LaTeX-Dokumentes. Auf Windows-Maschinen ist statt utf8 auch ANSIC als Codierung möglich, aber unnötig, da utf8 in jeder Hinsicht besser als ANSI ist. Bei Linux: latin1 als Codierung, auf MacOS X: applemac
\usepackage[T1]{fontenc}
\usepackage{gensymb}
\usepackage[english]{babel} %Deutsche Zeichen- und Umbruchsetzung
\usepackage{amsmath, amssymb,amsfonts} %AMS-TeX-Pakete. Nötig für die Definition der Mathematik-Umgebung
\usepackage{lmodern}
\usepackage{graphicx} %Nötig, um Grafiken einbinden zu können
\usepackage{geometry}
\usepackage{biblatex}
\addbibresource{lit.bib} %Literaturverzeichnis, zu finden als Datei lit.bib im selben Ordner wie diese .tex-Datei
\usepackage{float}
%Ein Paket, mit dem sich ohne Probleme mehrseitige PDF-Dokumente ohne \includegraphics-Rumgemache einbinden lassen. Befehl: \includepdf[pages=a-b]{PDFfile.pdf}. Einzelne Seiten, oder auch alle Seiten (Option pages=-) können angewählt werden
%Wenn keine Option angegeben wird, gilt pages=1!
\usepackage[final]{pdfpages}
\usepackage{framed, color} %Framed: Paket, mittels dessen ein Rahmen um einen Bereich definiert werden kann. Color: Lässt Farbdarstellung in Schrift, Hintergrund etc. zu
\usepackage{scrlayer-scrpage} %Header für die KOMA-script -Klasse
\usepackage{siunitx} %Ein schönes Paket, um Einheiten und physikalische Größen richtig zu setzen. Z.B.  \SI{2}{\kilo\gram\per\meter\squared}
%\sisetup{locale = US} %macht das die Punkte in Rechnungen Punkte bleiben
\usepackage{subfigure} %Mehrere Bilder in einer Figure-Umgebung
\usepackage{mhchem}
\usepackage[bookmarks,colorlinks=true]{hyperref} %Mittels hyperref lassen sich hyperlinks innerhalb des PDF-Dokumentes benutzen. Beispiel: Mausklick im Inhaltsverzeichnis auf ein Kapitel führt zum automatischen Sprung in dieses Kapitel
\graphicspath{{images/}}
%siunitx-Konfiguration. Damit werden die richtigen Font-Einstellungen erkannt (also beispielsweise fett, kursiv etc.) und damit  ebenfalls die deutsche Zeichensetzung, insbesondere Trennungszeichen, benutzt werden. 
%Ebenso wird bei SIrange das "`to"' in "`bis"' umgewandelt, bei SIlist das "`and"' in "`und"'
\sisetup{detect-weight=true, detect-family=true, locale=US, range-phrase={\,bis\,}, 
list-final-separator={\,\linebreak[0] \text{und}\,}, separate-uncertainty=true, 
per-mode=symbol-or-fraction}
%\SI[per-mode = fraction]{1}{\meter\per\second} erzwingt auch im Fließtext die Bruchdarstellung.
\DeclareSIUnit\curie{Ci}%Zusätzliche Einheit definieren

%Hyperlinks-Setup
\hypersetup{
	colorlinks,
	linktocpage,
	citecolor=black,
	filecolor=black,
	linkcolor=black,
	urlcolor=black
}

\numberwithin{equation}{section} % Die Nummerierung von Gleichungen bekommt die jeweilige Section-Nummer als Präfix

\setlength{\parindent}{0 mm} %Einrücktiefe von neuen Absätzen
\setlength{\parskip}{2 mm} %Abstand von Absätzen



\pagestyle{scrheadings}%Kopf und Fußzeilen
\ihead{\VerfasserEINS\;\&\;\VerfasserZWEI}
\ofoot{\thepage}
\cfoot{\empty}
\ifoot{\empty} 

\newcommand{\VERSUCHSDATUM}{10/27/2025}
\newcommand{\PROTOKOLLDATUM}{\today}

\newcommand{\VerfasserEINS}{Felix Roth}
\newcommand{\MatNoEINS}{3711192}
\newcommand{\StudiengangEINS}{B. Sc. Chemistry}
\newcommand{\stmailEINS}{st188157@stud.uni-stuttgart.de}

\newcommand{\VerfasserZWEI}{Jarno Adam}
\newcommand{\MatNoZWEI}{3725706}
\newcommand{\StudiengangZWEI}{B. Sc. Chemistry}
\newcommand{\stmailZWEI}{st188875@stud.uni-stuttgart.de}

\newcommand{\Assistent}{Anna Savchenko}

\newcommand{\VERSUCHSNAME}{Infrared Spectroscopy and Raman Spectroscopy}


\begin{document}
\thispagestyle{empty}
\begin{titlepage}
    \title{Abgabe Blatt 5}
    \author{Laurens Viehoff, Felix Roth}
    \date{\today}
\maketitle
\end{titlepage}


\thispagestyle{empty}
\tableofcontents 
\clearpage 

\renewcommand{\thepage}{\arabic{page}}
\setcounter{page}{1}

\section{Gleitumbegung Erklärung}

    LaTeX behandelt Elemente wie Bilder und Tabellen als Gleitobjekte, die automatisch an optisch passenden Stellen platziert werden, statt starr im Textfluss zu verbleiben. 
    Nutzer können diese Positionierung jedoch gezielt durch Parameter wie h (hier), t (top), b (bottom) oder sowie durch Anpassung interner Einstellungen steuern, um unerwünschte Verschiebungen zu vermeiden.

\section{Evaluation}

\subsection{Raman}
For the following Raman spectra, the selection rule applies, that the polarizability of the molecule needs to change with 
the absorption. This means only those vibrational modes are Raman active and will thus be visible in the spectrum, which are 
linked to a change in the dipole of the molecule. \cite{Script}
\subsubsection{Comparison of experimental and calculated spectra}
The calculated and meassured peak positions for the chloroform are listed in table \eqref{Theo_exp_Raman_Chloroform}
\begin{table}[H] 
    \centering
    \caption{Theoretically calculated peak positions $\tilde{\nu}_\text{theo}$ with their activity $A$ and depolarisation $\rho$ and experimentally meassured peak positions $\tilde{\nu}_\text{exp}$ with their intensitiy $I$ for chloroform.}
    \begin{tabular}{|c|c|c|c|c|c|}
        \hline
        Mode & $\tilde{\nu}_\text{theo}$ & $A$ & $\rho$ & $\tilde{\nu}_\text{exp}$ & $I$ \\
        \hline        
	6 &      254.65 & 5.14 & 0.75 & -  &- \\
   7  &     254.97  & 5.13 & 0.75 & 402.08 & 2913.67\\
   8  &    361.89   & 8.68 & 0.24 & 492.15 & 3156.67 \\
   9  &   665.44    & 9.80 & 0.00 & - & -\\
  10  &  740.53     & 3.09 & 0.75 & 759.03 & 3644.33 \\
  11  & 740.76      & 3.08 & 0.75 & 839.09  &    1774 \\
  12  &    1219.78  & 6.03 & 0.75 &  - & -\\
  13  &   1220.55   & 6.02 & 0.75 & 1266.11 & 1429 \\
  14  &  3168.76    & 77.21 & 0.24 & 3187.65 &  3136\\
         \hline
    \end{tabular}
    \label{Theo_exp_Raman_Chloroform}
\end{table}
It is clearly visible that the experimental data follow the Raman selection rules, as vibrational mode 9 is not observable. 
Additionally, deviations in the wavenumbers are seen between the simulated and experimental data. Furthermore, there are 
several simulated peaks that possess similar wavenumbers and are superimposed in the experimental data due to line 
broadening.
\\
The recorded spectrum for chloroform is depicted in figure \ref{fig:Raman_Chloroform}.
\begin{figure}[H]
	\centering
    \includegraphics[width=0.9\linewidth]{ramanChloroform.png}
    \caption{Raman spectrum of Chloroform.}
    \label{fig:Raman_Chloroform}
\end{figure}
In this plot, 8 peaks are visible, 4 of which at a wavenumber lower than \SI{1000}{cm^{-1}}, one at around \SI{1200}{cm^{-1}} 
and one at around \SI{3000}{cm^{-1}}. The one at \SI{2500}{cm^{-1}} must be caused by impurifications. The peak at 
\SI{2000}{cm^{-1}} is the Rayleigh peak which is present in all the following spectra.
\begin{equation}\label{dummeRechnung}
	\frac{\sqrt{\tilde{\mu_\text{H}}}}{\sqrt{\tilde{\mu_\text{D}}}} = \frac{\SI{0.923}{u}}{\SI{1.714}{u}} = 0.538
\end{equation}
Mit der Formel \eqref{dummeRechnung} kann das Verhältnis der der C-H zur C-D Schwinungung berechnet werden, wofür sich in 
diesem Bespiel ein Verhältnis von 0.539. 

\newpage
\listoffigures
\listoftables
\printbibliography
\end{document}
